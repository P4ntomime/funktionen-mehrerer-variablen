\section{Extrema von Funktionen (bi-variat)}


\subsection{Stationärbedingungen} % TODO: Ergänzen um Eigenwerte + Lagrange (evtl. statt Ergänzen, neuer Abschnitt)
\begin{outline}
    \1 Nicht am Rand des Definitionsbereichs ($\mathbb{D}_f$)
    \1 $f$ muss differenzierbar sein
\end{outline}

$f$ lokal (relativ) extremal \textrightarrow\ horizontale Tangentialebene


\subsubsection{Verallgemeinerung des Fermat-Prinzips}
\[
    \begin{array}{c}
        f(x_0; y_0) + \cub{red}{\diff f}{0} = f(x_0; y_0) + \cub{red}{f_x \diff x + f_y \diff y}{0}\\ % TODO: fix underbrace size
        \Leftrightarrow f_x \overset{!}{=} 0 \land f_y \overset{!}{=} 0 \Leftrightarrow \grad(f) = \vec{0}
    \end{array}
\]


\subsection{Hinreichende Bedingungen}
Aus An1: $\underset{\min}{f'' > 0}\, |\, \underset{\max}{f'' < 0}$

Ausnahme: Sattelfläche

\begingroup
\setlength{\arraycolsep}{1mm}
\begin{minipage}[t]{.35\columnwidth} % TODO: Eigenwerte ergänzen
    \[
        (v_1; v_2)
        \underbrace{\begin{pmatrix}
            f_{xx}                  &\tikznode{fxy}{$f_{xy}$}\\
            \tikznode{fyx}{$f_{yx}$}  &f_{yy}
        \end{pmatrix}}_{\text{Hess-Matrix ($H$)}}
        \begin{pmatrix}
            v_1\\
            v_2
        \end{pmatrix}
    \]
    \tikz[overlay, remember picture] {
        \node[anchor=center, inner sep=0pt, rotate=40] at ($(fxy)!0.55!(fyx)$) {$=$};
    }
\end{minipage}\hfill
\begin{minipage}[t]{.55\columnwidth}
    \[
        \Delta = \det(H) = f_{xx}f_{yy} - f_{xy}^2
    \]
    \begin{outline}
        \1[$\Delta > 0$:] Extremalstelle (max oder min)
        \1[$\Delta < 0$:] Sattel-Situation (\textbf{nicht} extremal)
        \1[$\Delta = 0$:] Unklar; Multi-Variate Taylor-Logik\\
            (wenn mögl. vermeiden)
    \end{outline}
\end{minipage}
\endgroup

% TODO: Weiszfeld-Algorithmus (nicht sicher, ob prüfungsrelevant) 
% TODO: Extremalwert mit Nebenbedingungen
% TODO: Lagrange-Funktion
% TODO: Geränderte Hess-Matrix
% TODO: Taylor Ordnung > 1
