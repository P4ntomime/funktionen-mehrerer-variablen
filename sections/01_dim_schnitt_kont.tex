\section{Dimensionen, Schnitte und Konturen}
\subsection{Dimensionen}
$$f: \mathbb{D}_f (\subseteqq  \rreal^{\cbl{\bm{m}}}) \longrightarrow \mathbb{W}_f (\subseteqq  \rreal^{\cgn{\bm{n}}})$$


\begin{ctabular}{O l}
    \cbl{\bm{m}} & Anzahl Dimensionen von ${\mathbb{D}_f}$, wobei $\cbl{\bm{m}}$ ${\in \nnatural }$\\
    \cgn{\bm{n}} & Anzahl Dimensionen von ${\mathbb{W}_f}$, wobei $\cgn{\bm{n}}$ ${\in \nnatural }$\\
    \vec{f} & wenn Output vektoriell
\end{ctabular}

\warn{Variablen sind abhängig von einander!}\\[2mm]

\textbf{Multi-Variat:}\\
Die Funktion ${f}$ ist "Multi-Variat", wenn:\\
- Input, Output oder beides mehrdimensional ist.\\
(Nur wenn Input und Output Skalare sind ist eine Funktion nicht Multi-Variat.)\\
%\example{Multi-Variat}
%\begin{ctabular}{O l}
%    (x;y) \mapsto \text{Skalar} \Rightarrow \text{Multi-Variat}\\
%    (x;y;z) \mapsto \text{Skalar} \Rightarrow \text{Multi-Variat}\\
%    (t;x;y;z) \mapsto \text{Skalar} \Rightarrow \text{Multi-Variat}\\
%\end{ctabular}

\subsubsection{Raumzeit}

\newenvironment{rcases}
  {\left.\begin{aligned}}
  {\end{aligned}\right\rbrace}

\begin{equation*}
    \begin{rcases}
      \text{Raum 3D ${(x;y;z)}$ }\rreal^3\\
      \text{Zeit 1D ${(t)}$ }\rreal^1
    \end{rcases}
    \rreal^1 \times \rreal^3 (t;x;y;z) = \text{4D Raumzeit}
\end{equation*}

\subsubsection{Stationärer Fall}
$${t \rightarrow \infty \rightarrow \text{Stationär}}$$
$${\crd{T(x;y;z) \enspace\frac{\Delta{T}}{\Delta{t}} \rightarrow 0 }}$$

\subsubsection{Koordinatenvektoren = Einheitsvektoren}
$\vec{i}=\hat{i}=\vec{{e}_1}=
\begin{pmatrix}
    1 \\
    0 \\
    0
\end{pmatrix},\enspace
\vec{j}=\hat{j}=\vec{{e}_2}=
\begin{pmatrix}
    0 \\
    1 \\
    0
\end{pmatrix},\enspace
\vec{k}=\hat{k}=\vec{{e}_3}=
\begin{pmatrix}
    0 \\
    0 \\
    1
\end{pmatrix}$

\subsection{Schnitte}
Schnitt = Restriktion $\rightarrow$ Teilmenge vom Definitionsbereich ${\mathbb{D}_f}$\\

\subsubsection{Partielle Funktion}
$\triangleright$ Nur EINE Variable ist frei! (wählbar)\\
$\triangleright$ ALLE anderen Variabeln sind fix!\\
\crd{$\triangleright$ \warn{$\mathbb{W}_f$ Analyse!}}\\[2mm]

\subsection{Konturen, Levelsets, Niveaulinien, ...}
Output = konstant = const. = fix: 
$$\vec{y} = \vec{f}(\vec{x}) = \text{const. wobei } \vec{x} \subset \mathbb{D}_f$$
Man spricht von "Konturen, Levelsets oder Niveaulinien",\\
wenn der Output von ${f}$ konstant ist.\\
\crd{Hier wäre ein Bild von Höhenlinien aus dem Skript cool}

