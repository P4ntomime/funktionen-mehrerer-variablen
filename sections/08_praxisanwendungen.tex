
\section{Anwendungen}

\subsection{Integralsatz von Gauss}
Der Integralsatz von Gauss 
\[
    \boxed{\int\limits_{V} \nabla \dotp \vec{A} \diff V = \oint\limits_{S = \partial V} \vec{A} \dotp \diff \vec{S}}
\]
beschreibt, dass die aufintegrierte Divergenz in einem Körper gleich dem Fluss durch die Kontur dieses Körpers sein muss.


\subsection{Integralsatz von Stokes}
Der Integralsatz von Stokes
\[
    \boxed{\int\limits_{S} \rot \vec{A} \dotp \hat{n} \diff S = \oint\limits_{C = \partial S} \vec{A} \dotp \diff \vec{r}}
\]
sagt aus, dass durch das Integrieren eines Vektorfelds $\vec{A}$ entlang der Kontur $C$ einer Fläche $S$ auf die mittleren Verwirbelungen im Innern der Fläche geschlossen werden kann.

Die Normale $\hat{n}$ und die Integrationsrichtung $\vec{r}$ müssen dabei die Rechte-Hand-Regel erfüllen.


\subsection{Poisson-Gleichung (Laplace-Gleichung)}
Die Poisson-Gleichung
\[
    \boxed{\Delta \phi (\vec{r})
    = f(\vec{r})}
    \hspace{1em}\text{oder}\hspace{1em}
    \boxed{\nabla^2 \phi (\vec{r})
    = f(\vec{r})}
\]
findet in der Physik oft Anwendung. $\phi$ beschreibt dabei ein skalares Potentialfeld, $f$ wird Quellenfunktion genannt und $\vec{r}$ ist ein beliebiger Stützvektor.

\subsubsection{Laplace-Gleichung}
Die Laplace-Gleichung
\[\boxed{\Delta \phi = f = 0}\] 
ist der Spezialfall der Poisson-Gleichung, bei dem keine Quellenfunktion $f$ besteht.


% TODO: Green'sche Funktion / Green'scher Satz
\subsection{Green'sche Funktion}
Die Green'sche Funktion (auch Green'scher Satz) ...


% TODO: Anwengungen: Maxwell-Gleichungen
\subsection{Maxwell-Gleichungen}
