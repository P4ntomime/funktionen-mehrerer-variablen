% ========================================= TEMPLATE INFO ========================================
%
% Author:       P4ntomime
% Version:      1.0.0
% Last updated: 2024-02-18
% Brief:        A LaTeX template for summaries. See README.md for more information.
% 
% ================================================================================================
\documentclass[8pt, a4paper, twoside]{extarticle}
% Font size:    8pt
% Paper size:   A4
% style:        twoside (needed, so odd and even pages have different margins)
% orientation:  portrait. (use 'landscape' for landscape orientation)


% ========================================= DOCUMENT INFO =========================================
\def\title{Funktionen mehrerer Variablen}                               % title
\def\shorttitle{FuVar}                                                  % short title (displayed as PDF title)
\def\dozent{Prof. Dr. Bernhard Zgraggen}                                % lecturer
\def\semester{FS 2024}                                                  % semester
\def\author{Laurin Heitzer, Flurin Brechbühler}                         % author(s)
\def\repo{https://github.com/P4ntomime/funktionen-mehrerer-variablen}   % repository link
\def\version{0.1.\today}                                                % version
\def\pagelimit{12}                                                      % page limit -> causes pages after limit to be red


% ================================= PACKAGES, SETUP AND COMMANDS ==================================
\input{preamble.tex}

% =========================================== DOCUMENT ============================================
\begin{document}
    \begin{layout}
        \section{Dimensionen, Schnitte und Konturen}

\subsection{Dimensionen}

$$f: \mathbb{D}_f (\subseteqq  \rreal^{\cbl{\bm{m}}}) \longrightarrow \mathbb{W}_f (\subseteqq  \rreal^{\cgn{\bm{n}}})$$


\begin{ctabular}{O l}
    \cbl{\bm{m}}    & Anzahl Dimensionen von ${\mathbb{D}_f}$, wobei $\cbl{\bm{m}}$ ${\in \nnatural }$\\
    \cgn{\bm{n}}    & Anzahl Dimensionen von ${\mathbb{W}_f}$, wobei $\cgn{\bm{n}}$ ${\in \nnatural }$\\
    \vec{f}         & wenn Output vektoriell
\end{ctabular}

\warn{Variablen sind abhängig von einander!}


\subsubsection*{Multi-Variat:}

Die Funktion ${f}$ ist "Multi-Variat", wenn:
\begin{itemize}
    \item Input, Output oder beides mehrdimensional ist.\\
    (Nur wenn Input \textbf{und} Output Skalare sind ist eine Funktion nicht Multi-Variat.)
\end{itemize}
%\example{Multi-Variat}
%\begin{ctabular}{O l}
%    (x;y) \mapsto \text{Skalar} \Rightarrow \text{Multi-Variat}\\
%    (x;y;z) \mapsto \text{Skalar} \Rightarrow \text{Multi-Variat}\\
%    (t;x;y;z) \mapsto \text{Skalar} \Rightarrow \text{Multi-Variat}\\
%\end{ctabular}


\subsubsection{Raumzeit}

$$\left.\begin{array}{r@{}}
    \text{Raum 3D ${(x;y;z)}$ }\rreal^3\\
    \text{Zeit 1D ${(t)}$ }\rreal^1
\end{array}\right\rbrace \rreal^{1}\times \rreal^3 (t;x;y;z) = \text{4D Raumzeit}$$


\subsubsection{Stationärer Fall}
$${t \rightarrow \infty \rightarrow \text{Stationär}}$$
$${\crd{T(x;y;z) \enspace\frac{\Delta{T}}{\Delta{t}} \rightarrow 0 }}$$


\subsubsection{Koordinatenvektoren = Einheitsvektoren}
$\vec{i}=\hat{i}=\vec{{e}_1}=
\begin{pmatrix}
    1 \\
    0 \\
    0
\end{pmatrix},\enspace
\vec{j}=\hat{j}=\vec{{e}_2}=
\begin{pmatrix}
    0 \\
    1 \\
    0
\end{pmatrix},\enspace
\vec{k}=\hat{k}=\vec{{e}_3}=
\begin{pmatrix}
    0 \\
    0 \\
    1
\end{pmatrix}$


\subsection{Schnitte}
Schnitt = Restriktion $\rightarrow$ Teilmenge vom Definitionsbereich ${\mathbb{D}_f}$


\subsubsection{Partielle Funktion}

\begin{itemize}
    \item Nur \textbf{eine} Variable ist frei! (wählbar)
    \item \textbf{Alle} anderen Variablen sind fix!
    \item[] \warn{$\mathbb{W}_f$ Analyse!}
\end{itemize}


\subsection{Konturen, Levelsets, Niveaulinien, ...}
Output = konstant = const. = fix: 
$$\vec{y} = \vec{f}(\vec{x}) = \text{const. wobei } \vec{x} \subset \mathbb{D}_f$$
Man spricht von "Konturen, Levelsets oder Niveaulinien",\\
wenn der Output von ${f}$ konstant ist.\\
\crd{Hier wäre ein Bild von Höhenlinien aus dem Skript cool}


        \newpage
\section{Ableitungen, DGL und Gradienten (bi-variat)}
\[
    f:\mathbb{D}_f\subseteq \mathbb{R}^2\rightarrow \mathbb{W}_f\subseteq \mathbb{R}\quad\mathrm{skalar}
\]
\subsection{Partielle Ableitung}
Ableitung einer Partiellen Funktion. 

\example{Bi-Variate Funktion}
\[
    f(x, y):\, y \text{ fixieren} = \text{const.} = y_{0};\quad x \text{ \textbf{einzige} freie Variable}
\]

\subsubsection*{Notationen}

\begin{ctabular}{l l}
    1. Ordnung: & $\displaystyle f(x; y_{0})\Rightarrow \frac{\partial f}{\partial x} = f_{x}(x; y_{0})$ \\
    \multirow{2}*{2. Ordnung:}  &  $\displaystyle \frac{\partial}{\partial x}
                                    \left\lgroup\frac{\partial f}{\partial x}\right\rgroup = 
                                    \frac{\partial^{2}f}{\partial x^{2}} = f_{xx}$\\
                                &  $\displaystyle \frac{\partial}{\partial y}
                                        \left\lgroup\frac{\partial f}{\partial x}\right\rgroup = 
                                    \frac{\partial^{2}f}{\partial y \partial x} = f_{xy}$
\end{ctabular}


\subsubsection{Schwarz-Symmetrie}
Wenn \cor{$f_{xx}, f_{yy}, f_{xy}$ \& $f_{yx}$} \textbf{stetig} (sprungfrei) sind, dann gilt:
\[
    f_{xy} \overset{!}{=} f_{yx}
\]

         
\subsection{Gradient (Nabla-Operator)}
Spaltenvektor mit partiellen Ableitungen
\[
    \tikznode{grad}{\nabla} f = \begin{pmatrix}
        \frac{\partial f^{\mathstrut}}{\partial^{\mathstrut} x}\\
        \frac{\partial f^{\mathstrut}}{\partial^{\mathstrut} y}\\
        \vdots      % TODO: reformat vdots. They look like shit
    \end{pmatrix} \hat{=} \text{Vektorfeld}
\]

% TODO: some TikZ. Might not be needed and could be removed if unnecessary
\begin{tikzpicture}[overlay, remember picture, >={Latex}]
    \node[inner sep=0pt, ellipse, fit=(grad), draw, densely dotted]{};
    \node[below left= 2mm and 5mm of grad] (gtext) {\txtqt{Gradient} / Nabla};
    \draw[->, rounded corners] (gtext.east) -| ($(grad.south)+(0,-1mm)$);
\end{tikzpicture}


\subsection{Totale Ableitung}
Für Fehlerrechnung benützt, da man hierbei die Abstände von $(x; y; z)$ 
zu einem festen Punkt $(x_{0}; y_{0}; z_{0})$ erhält. (relative Koordinaten)

\[
    D(f;\underbrace{(x_{0}, y_{0}, \ldots)}_{\text{Arbeitspunkt}}):\, 
    \rreal^{\tikznode{somea}{\scriptstyle 2}} \rightarrow \rreal^{\tikznode{someb}{\scriptstyle 1}};\, 
    \text{\txtqt{gute Approximation}}
\]
\[
    f(x = x_{0} + \Delta x; y = y_{0} + \Delta y; \ldots) = (D_{11}; D_{12})\cdot \begin{pmatrix}
        \Delta x\\
        \Delta y
    \end{pmatrix} + f(x_{0}; y_{0}) + \cgn{R_{1}}
\]
Wobei \cgn{$R_{1}$} dem \txtqt{Rest} entspricht. (Ähnlich wie bei Taylorreihe)

\begin{minipage}[c]{0.7\columnwidth}
    \[
        \frac{\cgn{R_{1}}}{\cor{d = \sqrt{\Delta x^{2} + \Delta y^{2}}}} \rightarrow 0 \text{ (\txtqt{gut}, \txtqt{schneller gegen 0 als $\cor{d}$})}
    \]
\end{minipage}\hfill
\begin{minipage}[c]{0.29\columnwidth}
    \begin{tikzpicture}[baseline=(current bounding box.center), 
                        >={Latex[width=1mm, 
                                 length=1mm]}, 
                        scale=0.5, 
                        font=\tiny]

        % Koordinatensystem
        \draw[->] (0,0) -- (4,0) node[below]{$x$};
        \draw[->] (0,0) -- (0,2.5) node[left]{$y$};

        % Punkte
        \node[circle, fill=black, inner sep=0pt, minimum size=2pt] (A) at (1.7,0.7) {};
        \node[circle, fill=black, inner sep=0pt, minimum size=2pt] (P) at (0.7,1.7) {};
        \node[inner sep=1pt, right=1mm of A] {$A=(x_{0}; y_{0})$};
        \node[inner sep=1pt, above right=0.1mm and -1ex of P] {$P=(x;y)$};
    
        % Differenz
        \draw (P) -- (A-|P) node[midway, left]{$\Delta y$} -- (A) node[midway, below]{$\Delta x$};
        \draw[orange] (P) -- (A) node[midway, above right, inner sep=1pt]{\cor{$d$}};
    \end{tikzpicture}
\end{minipage}

\[
    \begin{split}
        D(f;(x_{0}; y_{0})) &= \left\lgroup D_{11} = \frac{\partial f}{\partial x}(x_{0}; y_{0}); 
                                            D_{12} = \frac{\partial f}{\partial y}(x_{0}; y_{0})\right\rgroup\\
        &= (\nabla f)^{\tr} \text{ \cbl{wenn $\frac{\partial f}{\partial x}; \frac{\partial f}{\partial y}$ stetig bei $A$}}
    \end{split}
\]

\begin{tikzpicture}[overlay, remember picture]
    \draw[tips, -{Latex}] (someb) to[out=135, in=45, edge node={node[above=0mm]{\tiny $1\times 2$ Matrix}}] (somea);
\end{tikzpicture}


\subsection{Linearapproximation (Tangentialapproximation)}
\[
    f(x;y) \approx f(x_{0}; y_{0}) + D(f;(x_{0}; y_{0}))\cdot \begin{pmatrix}
        \Delta x\\
        \Delta y
    \end{pmatrix}
    \quad\text{ linear in $\Delta x$ und $\Delta y$}
\]


\subsubsection{Tangentialebene}
\[
    \crd{g(x; y) = f(x_{0}; y_{0}) + D(f;(x_{0}; y_{0}))\cdot \begin{pmatrix}
        x - x_{0}\\
        y - y_{0}
    \end{pmatrix}}
\]
\[
    g(x;y)=f(x_0;y_0)+f_x(x_0;y_0)\cdot(x-x_0)+f_y(x_0;y_0)\cdot(y-y_0)
\]

\subsubsection{Tangentialer Anstieg (Totale Differential)}
\[
    \cvt{\diff f \overset{!}{=} 
        \frac{\partial f}{\partial \tikznode{ptx}{x}}\diff x + 
        \frac{\partial f}{\partial \tikznode{pty}{y}}\diff y} 
        \quad \text{bezüglich } A=\underbrace{(x_{0}; y_{0})}_{\tikznode{axy}{}}
\]
\begin{tikzpicture}[overlay, 
                    remember picture, 
                    >={Latex[width=1mm, 
                             length=1mm]}]

    \draw[->] ($(axy)+(0, 1.8mm)$) to[bend left=10] (ptx.south east);
    \draw[->] ($(axy)+(0, 1.8mm)$) to[bend left=10] (pty.south east);
\end{tikzpicture}


\subsubsection{Differential-Trick (\texorpdfstring{$\diff f$}{df} Trick)}
\begin{minipage}[c]{0.5\columnwidth}
    \[\left\lgroup\begin{aligned}
        f &= c = \mathrm{const.} \quad | \diff(\ldots)\\
        \diff f &= \diff c \overset{!}{=} 0
    \end{aligned}\right\rgroup\]
\end{minipage}\hfill
\begin{minipage}[c]{0.5\columnwidth}
    \[
        f_{x}\diff x + f_{y}\diff y = 0 \quad \text{für Kontourlinien}
    \]
\end{minipage}


\subsubsection{Implizite (Steigungs-)Funktion}
\begin{minipage}[c]{0.6\columnwidth}
    \[
        \cbl{y'(x)} = \frac{\diff y}{\diff x} = -\frac{f_{x}}{f_{y}\crd{\neq 0}} \lor 
        \cbl{x'(y)} = \frac{\diff x}{\diff y} = -\frac{f_{y}}{f_{x}\crd{\neq 0}}
    \]
\end{minipage}\hfill
\begin{minipage}[c]{0.39\columnwidth}
    \begin{tikzpicture}[>={Latex[width=1mm,
                                 length=1mm]}, 
                        scale=0.75, 
                        font=\small]

        % Koordinatensystem
        \draw[->] (0,0) -- (3,0) node[below]{$x$};
        \draw[->] (0,0) -- (0,2) node[left]{$y$};
        
        % Funktionen
        \draw[color=green] plot[domain=0.55:1.5] (\x, {-((\x-1.5)*(\x-1.5))+1}) node[right]{$y$}; % y = -((x-1.5)^2)+1
        \draw[color=blue] node[above right]{$y'$} plot[domain=0.25:1.5] (\x, {\x-0.25}); % y = x-0.25

        % Arbeitspunkt
        \node[circle, fill=black, inner sep=0pt, minimum size=1.5pt] (P) at (1,0.75) {};
        \node[inner sep=0pt, above left=0.25mm and 0mm of P] {$P_{0}$};
        \draw[gray, dashed] (P) -- (P-|0,0) node[left]{$y_{0}$};
        \draw[gray, dashed] (P) -- (P|-0,0) node[below]{$x_{0}$};

        % Richtungselemente
        \draw[orange, semithick, ->] (P) -- (2,1.75) node[above, midway, rotate=45]{$\vec{r}$};
        \draw[->] (P) -- (P-|2,0) node[midway, below]{$\diff x$};
        \draw[->] (P-|2,0) -- (2,1.75) node[midway, right]{$\diff y$};
    \end{tikzpicture}
\end{minipage}




% \subsection{Differential}


\subsection{DGL}
\[
    y' = \tikznode{rhsfn}{\bbr{violet}{-\frac{f_{x}}{f_{y}}}};\, y(x_{0}) = y_{0}
\]
\begin{tikzpicture}[overlay, 
                    remember picture] 
    \node[below=1mm of rhsfn, inner sep=0pt, font=\tiny, text=violet] {right-hand-side (r.h.s.) Funktion};
\end{tikzpicture}


\subsection{Richtungselement (Tangentiallinie an Kontouren)}
\[
    \vec{r} = \left\lgroup\diff x = h; \diff y = y'\diff x = -\frac{f_{x}}{f_{y}}\diff x\right\rgroup^{\tr}
\]


\subsection{Gradientenfeld \texorpdfstring{$\perp$}{\_|\_} Kontouren}
\[
    \nabla f \tikznode{scprd}{\dotp} \begin{pmatrix}
        \diff x\\
        \diff y = y'\diff x
    \end{pmatrix} \overset{!}{=} 0
\]
\begin{tikzpicture}[overlay,
                    remember picture,
                    >={Latex[width=1mm, 
                             length=1mm]}]
    \node[above left=2mm and 4mm of scprd, inner sep=0pt, anchor=south east] (spnode) {Skalarprodukt};
    \draw[->] (spnode.east) to[out=0, in=90] ($(scprd.north)+(0,0.2mm)$);
\end{tikzpicture}


\subsection{?Wie heisst dieser Abschnitt?}

\begin{tikzpicture}[>={Latex[width=1mm,
                       length=1mm]}]
    \draw[->] (-1,0) -- (3,0) node[below]{$x$};
    \draw[->] (0,-0.5) -- (0,2) node[left]{$y$};

    % Schnitt
    \draw[green] (-1,-0.5) -- (1, 0.75);
\end{tikzpicture}


\begin{ctabular}{r l}
    $s(t) :$    &$P_0 + t \cdot \hat{v} \mid t\in \rreal$\\
    $s(t) :$    &$f(x_0 + t \cdot \hat{v}_1\, ;\, y_0 + t \cdot \hat{v}_2)$\\
    &\\
    $\frac{\diff s(t)}{\diff t}=\dot{s}(t) :$&$t\mapsto\overbrace{\begin{pmatrix}x_0+t\cdot v_1\\y_0+t\cdot v_2\end{pmatrix}}^{\begin{pmatrix}x\\y\end{pmatrix}}\mapsto f(x,y)$
\end{ctabular}


\subsection{Richtungs-Ableitung}
\[
    \frac{\partial f}{\partial\hat{v}}\overset{!}=D(f;(x_{0};y_{0}))\cdot\hat{v}\overset{\text{Def.}}\Leftrightarrow \grad(f)^{\tr}\cdot\hat{v}=f_{x}\cdot v_{1}+f_{y}\cdot v_{2}
\]

\example{Richtungs-Ableitung}

\[
    \vec{x} : \vec{v} = 
\begin{pmatrix}
    1\\
    0
\end{pmatrix}
= \hat{e}_1
\quad\Rightarrow\quad
    \frac{\partial f}{\partial\hat{e}_{1}}=f_{x}\cdot1+f_{y}\cdot0=\uuline{f_{x}\strut}
\]


\subsubsection{Spezialfälle}


\begin{minipage}[c]{.72\columnwidth}
    \begin{outline}
        \1 $\alpha = \frac{\pi}{2}$ \textrightarrow\ rechter Winkel % TODO: fix \pi --> should be upright
        \1 $\frac{\partial f}{\partial \hat{v}}$ extremal
            \2 $\alpha = 0$ (max):   $\nabla f \cdot\hat{v} > 0$ \textrightarrow\ \cbl{$\grad(f)$} liegt auf $\hat{v}$
            \2 $\alpha = \pi$ (min): $\nabla f \cdot\hat{v} < 0$ \textrightarrow\ \cbl{$\grad(f)$} liegt invers auf $\hat{v}$
    \end{outline}
\end{minipage}\hfill
\begin{tikzpicture}[%
    baseline=(current bounding box.center),
    scale=0.75,
    >={Latex[%
        width=1mm, 
        length=1mm]}]
    \coordinate (A) at (0,0);
    \coordinate (B1) at (2,0);
    \coordinate (B2) at (2.4,0);
    \coordinate (C) at (2,0.8);

    \draw[->] (A) -- (B2) node[right]{$\hat{v}$};
    \draw[blue, ->] (A) -- (C) node[above=-1mm]{$\grad(f)$};
    \draw[red, ->] (A) -- (B1) node[midway, below]{$\nabla f \dotp \hat{v}$};

    \draw[gray, densely dashed] (C) -- (B1);

    \draw pic[draw=green, angle radius=10mm, pic text=$\alpha$, pic text options=green] {angle=B1--A--C};
    \draw pic[draw=gray, angle radius=2mm, pic text=. , pic text options=gray] {angle=C--B1--A};

    \fill (A) circle (1pt) node[above]{$P_0$};
\end{tikzpicture}

\para{Trigo} $\nabla f\cdot\hat{v} \land \frac{\partial f}{\partial\hat{v}}$ \textrightarrow\ $\cos(\alpha)\cdot\abs{\cbl{\nabla f}}$


        \newpage
\section{Extrema von Funktionen zweier Variabeln finden}

\begin{enumerate}[itemsep=1ex]
    \item \textbf{Gradient von $f$ Null-setzten und kritische Stellen finden:}

    $\nabla f=
    \begin{pmatrix}
        f_x\\
        f_y
    \end{pmatrix} \stackrel{!}{=}
    \begin{pmatrix}
        0\\
        0
    \end{pmatrix}
    \, \, \, \, \, \,
    \Rightarrow 
    \begin{matrix}
        f_{x}=0\\
        f_{y}=0
    \end{matrix}
    \, \, \, \, \, \,
    \Rightarrow
    x_0 \text{ und } y_0 \text{ bestimmen}$

    \item \textbf{Zweite Partielle Ableitungen bestimmen:}
    
    $\begin{aligned}
        f_{xx} &= \dots\\
        f_{xy} &= f_{yx} = \dots\\
        f_{yy} &= \dots
    \end{aligned}$
    

    \item \textbf{Determinante $\Delta$ der Hesse-Matrix H bestimmen:}
    
    $\Delta = f_{xx}(x_0;y_0) \cdot f_{yy}(x_0;y_0) - \left(f_{xy}(x_0;y_0)\right)^2 $

    \item \textbf{Auswertung:}
    
    \begin{tabular}{lllcl}
        \hline
        $\Delta > 0$ &AND& $f_{xx}(x_0;y_0) < 0$ &$\Longrightarrow$& $\text{lokales Maximum}$\\
        \hline
        $\Delta > 0$ &AND& $f_{yy}(x_0;y_0) < 0$ &$\Longrightarrow$& $\text{lokales Maximum}$\\
        \hline
        $\Delta > 0$ &AND& $f_{xx}(x_0;y_0) > 0$ &$\Longrightarrow$& $\text{lokales Minimum}$\\
        \hline
        $\Delta > 0$ &AND& $f_{yy}(x_0;y_0) > 0$ &$\Longrightarrow$& $\text{lokales Minimum}$\\
        \hline
        $\Delta < 0$ &&&$\Longrightarrow$& $\text{Sattelpunkt}$\\
        \hline
        $\Delta = 0$ &&&?& $\text{Multi-variate-Taylor-logik ...}$\\
        \hline
    \end{tabular}

\end{enumerate}


        \section{Ableitungen, Extrema (multi-variat)}

        \section{Integration (bi-variat)}

\subsection{2D}

\[
    \int\int\limits_{\Omega}f(x;y)\cdot \diff x\cdot \diff y=\int\limits_{X}\Bigg(\int\limits_{Y}f(x;y)\cdot dy \Bigg)\cdot dx
\]

\[
    wenn\int\int|f(x;y)|dxdy<\infty
\]

\subsection{Normalbereich}

Schnitte sind Strecken (Intervalle) für x, y, ...

\subsection{Polar}

\[
    \text{d}x \cdot \text{d}y = r \cdot d\phi \cdot dr
\]

\subsection{2D Transformation Polar zu Kartesisch}
T $=$ Transformation
\[
    \text{Polar } (r,\phi) \xrightarrow{T} (x,y) \text{ Kartesisch}
\]

\[
\begin{pmatrix}
    x=r\cdot\cos(\varphi) \text{ } \cor{\mathbb{R}} \\
    y=r\cdot\sin(\varphi) \text{ } \cor{\mathbb{R}} 
\end{pmatrix}
\text{2D}
\]

Die Funktionen für $x$ und $y$ sind skalare Funktion.

    \begin{ctabular}{ll}
        $x=x(r;\varphi)$ & $ y=y(r;\varphi)$
    \end{ctabular}

\subsection{Derivative, Ableitung}


\subsection{}
\subsection{}
\subsection{}
        % TODO: Koordinatentransformation (Längenverzerrung, Elementverzerrung, Längenelement)
        % TODO: Fubini (Integration)
        % TODO: Transformation / Jacobi
        \newpage
\section{Integration (multi-variat)}

% TODO: Koordinatentransformation (Längenverzerrung, Elementverzerrung, Längenelement)


        % TODO: Fubini (Integration)
        % TODO: Transformation / Jacobi
        \section{Differenziation und Integration von Kurven}
\subsection{Kurvenintegral 1. Art}
Mit dem Kurvenintegral 1. Art wird die Länge einer Kurve in einer Ebene oder im Raum bestimmt.

\subsubsection{Zweidimensional}
Um die Länge einer Kurve $K$, die durch $f(x, y)$ beschrieben wird, in der Ebene zu bestimmen, wird über das Linienelement $ds$ integriert:
\[
    \iint\limits_{K} ds = \int_{x_1}^{x_2} \int_{y_1}^{y_2} \sqrt{dx^2 + dy^2}
\]
Dabei ist es nötig, die Funktion $f(x)$ in die Parameterdarstellung $f(x(t), y(t))$ zu bringen, da der Ausdurck $\sqrt{dx^2 + dy^2}$ problematisch ist.
Es folgt
\[
    \int_{x_1}^{x_2} \int_{y_1}^{y_2} f(x, y) \sqrt{dx^2 + dy^2} \frac{dt}{dt} = \int_{t_0}^{T} \sqrt{(\frac{dx}{dt})^2 + (\frac{dy}{dt})^2} dt,
\]
wobei $\frac{dx}{dt}$ und $\frac{dy}{dt}$ Funktionen sind, die durch Ableiten von $x(t)$ bzw. $y(t)$ nach $t$ berechnet werden können.

\subsubsection{Dreidimensional}
Das Kurvenintegral 1. Art in drei Dimensionen wurde bereits in Kapitel \ref{section:int_multivar:länge_einer_fkt} beschrieben.
% TODO: evtl. hier hin verschieben

\subsection{Kurvenintegral 2. Art}
Beim Kurvenintegral 2. Art wird nicht die tatsächliche Länge einer Funktion, sondern die Länge deren Projektion auf eine Achse bestimmt.
Dazu wird stat über alle Koordinatenrichtungen nur über eine der Koordinaten integriert.

Es folgen einige Paare von Kurvenintegralen 2. Art entlang einer Kontur $K$ für Funktionen in expliziter Form und in Parameterdarstellung.

\myul{2D, Projektion auf x:}
\[
    \int\limits_{K}f(x, y)dx = \int_{t_0}^{T}f(x(t), y(t)) \cdot x'(t) \cdot dt
\]

\myul{3D, Projektion auf x:}
\[
    \int\limits_{K}f(x, y, z)dx = \int_{t_0}^{T}f(x(t), y(t), z(t)) \cdot x'(t) \cdot dt
\]

\subsubsection{Anwendungen}
TODO: Für was wird das gebraucht?!

        % TODO: Kurvenintegral 2. Art
        \input{sections/08_ober-flächenintegrale.tex}
        % TODO: Allgemeine Wendelfläche
        % TODO: Freie Fläche (parametrisiert)
        % TODO: 1. metrischer Tensor
        
        \section{Vektoranalysis}
% TODO: Skalarer Durchfluss (?)
% TODO: Spezialfälle (?)
% TODO: Vektor Längenelement (?)

% TODO: Vektorfelder
\subsection{Vektorfelder}
\begin{itemize}
    \item Jedem Punkt $P$ im Raum ist ein Vektor $\vec{V}$ zugeordnet
    \item Kann als $\vec{V}(\vec{r})$ geschrieben werden, wobei $\vec{r}$ ein Ortsvektor mit fixem Ursprung $\vec{0}$ ist
    % \item Kann auch als Gradient eines Skalarfeldes $\phi$ geschrieben werden: $\vec{V} = \nabla \phi$. % TODO: bei bedarf einfügen
\end{itemize}


% TODO: Divergenz (Volumenableitung) (Cartesisch; Nabla / del-Operator)
\subsection{Divergenz (Volumenableitung)}
\begin{outline}
    \1 Beschreibt, wie stark sich ein Vektorfeld in einem Punkt ausbreitet oder zusammenzieht
    \1 Beispiel: Vektorfeld das die Geschwindigkeit von Wasser in eineem Fluss beschreibt
        \2 An Punkten mit positiver Divergenz fliesst Wasser hinaus (Quelle)
        \2 An Punkten mit negativer Divergenz fliesst Wasser hinein (Senke)
\end{outline}

$\boxed{\nabla\vec{V} = \div \vec{V} = \lim_{\Delta V\to 0}\frac{\oint_{\scriptscriptstyle (S)}\vec{V}\cdot\diff\vec{S}}{\Delta V}}$


\subsubsection{Kartesisch}
\[
    \boxed{%
        \div \vec{V}
        = \nabla\cdot\vec{V}
        = \underbrace{%
            \left\lgroup%
                \frac{\partial}{\partial x};\frac{\partial}{\partial y};\frac{\partial}{\partial z}%
            \right\rgroup}_{\nabla} \cdot 
        \begin{pmatrix}
            V_x \\ V_y \\ V_z
        \end{pmatrix}
        = \frac{\partial V_x}{\partial x} + \frac{\partial V_y}{\partial y} + \frac{\partial V_z}{\partial z}
    }
\]


\subsubsection{Zylinderkoordinaten}
\[
    \div \vec{V} = \frac{1}{r} \frac{\partial}{\partial r} (rV_r) + \frac{1}{r} \frac{\partial V_\varphi}{\partial \varphi} + \frac{\partial V_z}{\partial z}
\]

% TODO: doublecheck this!!
% \subsubsection{Kugelkoordinaten}
% \[
%     \frac{1}{r^2}\frac{\partial}{\partial r} (r^2 V_r) + \frac{1}{r\sin\vartheta}\frac{\partial}{\partial \vartheta} (\sin\vartheta V_\vartheta) + \frac{1}{r\sin\vartheta}\frac{\partial V_\varphi}{\partial \varphi}
% \]



% TODO: Poisson-Gleichung (Laplace-Gleichung) -- two subsections?
\subsection{Poisson-Gleichung (Laplace-Gleichung)}


$\boxed{\Delta \phi
    = \div\left\lgroup\grad(\phi)\right\rgroup
    = \nabla^2 \phi
    = \frac{\partial^2 \phi}{\partial x^2} + \frac{\partial^2 \phi}{\partial y^2} + \frac{\partial^2 \phi}{\partial z^2}
    = f(\vec{r})}$
\begin{tabular}{O<{:} l}
    \Delta & Laplace-Operator\\
    \phi & Potentialfeld\\
    f(\vec{r}) & Quellfunktion
\end{tabular}

\subsubsection{Laplace-Gleichung}
$\boxed{\Delta \phi = f = 0}$ \textrightarrow\ Spezialfall der Poisson-Gleichung ohne äussere Quellfunktion

% TODO: Green'sche Funktion / Green'scher Satz
% TODO: Beispiel Poisson-Gleichung


% TODO: Rotation eines Vektorfelds / Rotationsfeld (rot(); curl)
\subsection[Rotation eines Vektorfelds (rot(), curl())]{Rotation eines Vektorfelds ($\rot()$, $\curl()$)}
\[
    \boxed{%
        \rot \vec{A}
        = \nabla \times \vec{A}
        =   \begin{pmatrix}
                \frac{\partial}{\partial x}\\
                \frac{\partial}{\partial y}\\
                \frac{\partial}{\partial z}
            \end{pmatrix} \times 
        \begin{pmatrix}
            A_x \\ A_y \\ A_z
        \end{pmatrix} =
        \frac{\partial A_x}{\partial x} + \frac{\partial A_y}{\partial y} + \frac{\partial A_z}{\partial z}
    }
\]

Gauss: $\div \rot(\vec{A}) \overset{!}{=} 0$

% TODO: Stokes Integralsatz % TODO: evtl als subsubsection?
\subsection{Stokes Integralsatz}


% TODO: Anwengungen: Maxwell-Gleichungen
\subsection{Anwendungen: Maxwell-Gleichungen}
% TODO: Koordinatensysteme (Kartesisch, Polar, Kugel (Geografisch & Math.))


        % TODO: Koordinatensysteme (Kartesisch, Polar, Kugel (Geografisch & Math.))
        % TODO: Vektorfelder
        % TODO: Skalarer Durchfluss
        % TODO: Spezialfälle
        % TODO: Vektor Längenelement
        % TODO: Divergenz (Volumenableitung) (Cartesisch; Nabla / del-Operator)
        % TODO: Poisson-Gleichung (Laplace-Gleichung)
        % TODO: rot()
        % TODO: Stokes Integralsatz
    \end{layout}
\end{document}
